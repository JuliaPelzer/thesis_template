%##########################################################################
%                                               
%						 Latex-Vorlage des VuV
%
%            für Studienarbeiten
%			 Autor: Julia Pelzer, 2021
%
%##########################################################################

%##########################################################################
% Formatierungsoptionen

\documentclass[12pt,a4paper,twoside]{report}
\usepackage{ucs}
\usepackage[utf8]{inputenc}
\usepackage[T1]{fontenc}
\usepackage{lmodern} %falls die Schriftdarstellung bei Kompilierung unter Windows % schlecht aussieht, dieses auskommentieren
%\usepackage[german]{babel} %auskommentieren, wenn englisches Dokument
\usepackage[english]{babel} %einkommentieren, wenn englisches Dokument
\usepackage{amsmath,amssymb,amsthm}
\usepackage{a4}
\usepackage{graphicx}
\usepackage{algorithmic,algorithm}
\usepackage{subfigure}

\usepackage{uarial} %Schriftart
\usepackage[a4paper,lmargin={2cm},rmargin={2.5cm},tmargin={2.5cm},bmargin = {2.5cm}]{geometry} %Ergänzung für Anforderungen VuV an Seitenformat
\linespread{1.2} %Ergänzung durch Anforderung VuV Zeilenabstand
\usepackage[colorlinks,pdfpagelabels,pdfstartview = FitH,bookmarksopen = true,bookmarksnumbered = true,linkcolor = black,plainpages = false,hypertexnames = false,citecolor = black]{hyperref}
\usepackage{caption}
\usepackage[singlelinecheck=false, justification=RaggedRight, format=plain]{caption} %Anpassung der Bild- und Tabellenbeschriftungen
% ############################################################################
% Inverse Suche mit xdvi und kile:
\usepackage{srcltx}

% Seitenstil
\pagestyle{headings}

% Abstand zwischen Abs"atzen
\setlength{\parskip}{1.5ex}

% Einr"uckung der ersten Zeile eines Absatzes unterdr"ucken
\setlength{\parindent}{0pt}

% Grosszuegigere Wortabstaende
\sloppy

% Damit Bilder m"oglichst da sind, wo man sie will
\setcounter{topnumber}{20}
\setcounter{bottomnumber}{20}
\setcounter{totalnumber}{20}
\renewcommand{\topfraction}{.9999}
\renewcommand{\bottomfraction}{.9999}
\renewcommand{\textfraction}{0}

%###########################################################################
% Bearbeitung von einzelnen Kapiteln

%\includeonly{bericht1Introduction}
%###########################################################################

\begin{document}

%###########################################################################
%
%   Titlepage
%
%###########################################################################
\begin{titlepage}
\begin{flushright}
  \vspace{10mm} 
         {\large \bf \hspace{13mm} \fbox{ Masterthesis No. XX }\\}
  \vspace{10mm}
         {\Large \bf \hspace{20mm}Title \\} 
         {\Large \bf \hspace{20mm}of the\\}
         {\Large \bf \hspace{20mm}thesis\\}
  \vspace{5mm}
         {\large \bf \hspace{20mm}German\\} 
         {\large \bf \hspace{20mm}Title\\} 
  \vspace{20mm}
         {\fbox{may insert graphic here}\\} %if no graphic is insert, please comment out this line with "%"
  \vspace{50mm}
  		 {\small Author: Max Mustermann}\\ %own name
  		 {\small Study program: XX}\\
  \vspace{10mm}
  		 {\small Co-supervisor: M.Sc. Magdalena Schilling}\\ %insert the name of the co-supervisor here
  		 {\small Supervisor: Prof. Dr.--Ing. Markus Friedrich}\\ %insert the name of the supervisor here
  \vspace{10mm}
         {\bf 01.01.2021}\\ %insert date of submission here
  \vspace{20mm}
\end{flushright}
\begin{flushleft}
\begin{minipage}{0.2\textwidth}
	\begin{flushright}
	{\includegraphics[scale=0.35]{Pictures/ISV_logo_weiss.png}}\\
	\hfill
	\end{flushright}
\end{minipage}
\begin{minipage}{0.75\textwidth}
         {\small submitted to} \\
         {\bf Universität Stuttgart} \\
         {\bf Institut für Straßen- und Verkehrswesen}\\
         {\bf Lehrstuhl für Verkehrsplanung und Verkehrsleittechnik}\\
\end{minipage}
\end{flushleft}
\end{titlepage}

\clearpage
\thispagestyle{empty}
%\cleardoublepage % include for English version
%%###########################################################################
%
%   Titelseite
%
%###########################################################################
\begin{titlepage}
\begin{flushright}
  \vspace{10mm} 
         {\large \bf \hspace{13mm} \fbox{ Masterarbeit Nr. XX }\\}
  \vspace{10mm}
         {\Large \bf \hspace{20mm}Titel \\} 
         {\Large \bf \hspace{20mm}der\\}
         {\Large \bf \hspace{20mm}Arbeit\\}
  \vspace{5mm}
         {\large \bf \hspace{20mm}English\\} 
         {\large \bf \hspace{20mm}Title\\} 
  \vspace{20mm}
         {\fbox{hier kann eine Grafik eingefügt werden}\\} % falls keine Grafik eingefügt wird, bitte die Zeile auskommentieren
  \vspace{50mm}
  		 {\small BearbeiterIn: Max Mustermann}\\ %hier eigenen Namen einfügen
  		 {\small Studiengang: XX}\\
  \vspace{10mm}
  		 {\small BetreuerIn: M.Sc. Magdalena Schilling}\\ %hier Betreuer einfügen
  		 {\small PrüferIn: Prof. Dr.--Ing. Markus Friedrich}\\ %hier Prüfer einfügen
  \vspace{10mm}
         {\bf 01.01.2021}\\ %hier das Abgabedatum einfügen
  \vspace{20mm}
\end{flushright}
\begin{flushleft}
\begin{minipage}{0.2\textwidth}
	\begin{flushright}
	{\includegraphics[scale=0.35]{Pictures/ISV_logo_weiss.png}}\\
	\hfill
	\end{flushright}
\end{minipage}
\begin{minipage}{0.75\textwidth}
         {\small vorgelegt an der} \\
         {\bf Universität Stuttgart} \\
         {\bf Institut für Straßen- und Verkehrswesen}\\
         {\bf Lehrstuhl für Verkehrsplanung und Verkehrsleittechnik}\\
\end{minipage}
\end{flushleft}
\end{titlepage}

\clearpage
\thispagestyle{empty}
%\cleardoublepage % include for German version

%###########################################################################
%
%   deutsches Abstract
%
%###########################################################################
\section*{Abstract/ Summary}

\textit{Dieses Dokument und das Inhaltsverzeichnis sind als Formatvorlage zu verstehen und nicht als allgemeingültiger Vorschlag zur inhaltlichen Gliederung.}

\textit{Hinweise zur inhaltlichen Ausgestaltung einer wissenschaftlichen Arbeit finden sich im StyleGuide des Lehrstuhls (auch über die Homepage verfügbar).}
\thispagestyle{empty}

\cleardoublepage % Diese Erklärung steht immer auf der rechten Seite
\section*{Declaration on Autonomy}
I hereby declare that I wrote this master thesis independently and did not make use of any support or sources other than those mentioned in the paper.

\vspace*{2cm}
\begin{center}
   \begin{minipage}{0.3\textwidth}
      \centering
      \rule{4cm}{0.2mm}\\
      Stuttgart, Date
   \end{minipage}\hfill
   \begin{minipage}{0.6\textwidth}
      \centering
      \rule{8cm}{0.2mm}\\
      Signature
   \end{minipage}
\end{center}
\vfill
\thispagestyle{empty} % include for English version
%\cleardoublepage % Diese Erklärung steht immer auf der rechten Seite
\section*{Selbstständigkeitserklärung}
Hiermit erkläre ich, dass ich die vorliegende Arbeit eigenständig verfasst habe und keine anderen Hilfestellungen oder Quellen als die angegebenen in Anspruch genommen habe.

Insbesondere habe ich keinen bezahlten Dienst mit der Anfertigung der gesamten Arbeit oder Teilen der Arbeit beauftragt.

\vspace*{2cm}
\begin{center}
   \begin{minipage}{0.3\textwidth}
      \centering
      \rule{4cm}{0.2mm}\\
      Ort, Datum
   \end{minipage}\hfill
   \begin{minipage}{0.6\textwidth}
      \centering
      \rule{8cm}{0.2mm}\\
      Unterschrift
   \end{minipage}
\end{center}
\vfill
\thispagestyle{empty} % include for German version

% Inhalt auf der rechten Seite beginnen
\thispagestyle{empty}\cleardoublepage
% Raendereinstellungen fuer Doppelseitigen Ausdruck
%\evensidemargin=2pt
%\oddsidemargin=40pt

\pagenumbering{roman}
\include{berichttoc}\cleardoublepage % automatically generated table of contents

\pagenumbering{arabic}
%###########################################################################
%
%   Einleitung
%
%###########################################################################
\chapter{Introduction}
Text Text

\section{Introduction into the usage of LaTeX}

A \textbf{picture} is included like Figure~\ref{fig:Pic1}.
\begin{figure}[hbtp] % hbtp tells Latex where to position the figure
\centering % centers the picture horizontally
\includegraphics[scale=0.5]{Pictures/isv_logo_weiss.png} %Put all pictures in one folder
\caption[Exemplary picture]{Exemplary picture: The logo of the ISV institute.} %with [] a different description of the picture in the list of figures can be achieved than in the caption
\label{fig:Pic1} % by this name the picture can be referenced in the text via \ref{}
\end{figure}

\textbf{Tables} are included like Table~\ref{tab:tablename}.
\begin{table}[htbp]
\caption{Exemplary table}
\label{tab:tablename} % by this name the table can be referenced in the text via \ref{}
\centering %centers the table horizontally
\begin{tabular}[c]{l|l} % define here, how many columns there should be and whether the content shound be on the left ("l"), centered ("c"), on the right ("r") or with automatic line breaks ("p{width}")
\hline
\textbf{German} & \textbf{English (GB)} \\ \hline
Nutzenfunktion & utility function \\
Umlegung & apportionment \\
\end{tabular}
\end{table}

An \textbf{equation} is included like Equation~\ref{eq:test}
\begin{align}
f(x) = c_1 \cdot x^2 + c_2 \cdot x + c_3
\end{align}\label{eq:test}
with parameters $c_1$, $c_2$ and $c_3$ and variable $x$.


A \textbf{source} can be cited by~\cite{NachnameNachname22016} (Take a look at the .tex file). To update the references in the pdf (optical sign that it is necessary: you see a [?] where the source is supposed to be), in TexMaker between the two arrows select BibTex and press run (the left arrow), then PDFLaTeX, then BibTex, then PDFLaTeX, then Schnelles Übersetzen, to see the result. It thereby automatically appears in the Bibliography chapter.
To activate \textit{autocomplete} for the available sources from your .bib file (created with Jabref or similar software), go to Bearbeiten -> Literaturverzeichnis aktualisieren.
%###########################################################################
%
%   Kapitel 2
%
%###########################################################################
\chapter{Basics}
Text Text
%###########################################################################
%
%   Kapitel 3
%
%###########################################################################
\chapter{Analysis}
Text Text
%###########################################################################
%
%   Zusammenfassung
%
%###########################################################################
\chapter{Summary}
Text Text
%###########################################################################
%
%   Anhang
%
%###########################################################################
\begin{appendix}
\chapter*{Appendix}
\setcounter{chapter}{1}
\addcontentsline{toc}{chapter}{Anhang}

%###########################################################################
%   Anhang A
%###########################################################################
\section{Appendix 1 Name}
\markboth{Anhang}{Anhang}
Text

%###########################################################################
%    Anhang B
%###########################################################################
\section{Appendix 2 Name}
\markboth{Anhang}{Anhang}
Text

%###########################################################################
\end{appendix}

\listoffigures % automatically generated
\listoftables % automatically generated

\bibliographystyle{bib_style_file_VuV} %TODO gibts ne englische Version? was tut das?
\bibliography{Own_Library}
% \include{berichtbib} % if sources should be added manually, then insert them in berichtbib instead of Own_Library (not recommended!)

\end{document}

%##########################################################################
